\documentclass[11pt,a4paper,addpoints]{exam}
\usepackage[T1]{fontenc}%Codificação
\usepackage[utf8]{inputenc}%Inclui palavras com acento
\usepackage[portuges]{babel,varioref}
\usepackage{amsmath}
\usepackage{amsfonts}
\usepackage{amssymb}

\noprintanswers
%\printanswers

\pointsinmargin 
%\pointsinrightmargin
%\bracketedpoints
\boxedpoints
%\noboxedpoints

\parindent=0pt 
\title{ESCOLAS SECUNDÁRIA DE CASA \vspace{-1em}}
\author{Departmento de Ciências Socias e Humanas\\Turma A - 5º ano -  Matemática C\\MAT1 Revisões \qquad agosto 20, 2020}
\date{\vspace{-5ex}}

\begin{document}
\maketitle
\begin{center}
  \fbox{\fbox{\parbox{6.5in}{\centering
        Responda às perguntas nos espaços fornecidos. Se ficar sem espaço, continue no verso da página.}}}
\end{center}
\makebox[\textwidth]{Nome e número:\enspace\hrulefill}

\begin{questions}

\question Utilizando os algarismos: 3, 4, 6 e 7 escreve:

\begin{parts}
  \part[5] o menor número ímpar de dois algarismos diferentes.
    \begin{solutionordottedlines}[1cm]
    37
  \end{solutionordottedlines}
  \part[5] o maior número par de três algarismos diferentes.
  \begin{solutionordottedlines}[1cm]
    764
  \end{solutionordottedlines}
\end{parts}

\question[10] Diz, justificando, se a afirmação seguinte é verdadeira ou falsa.
“Os números 2, 46, 820 e 954 são todos números pares!”
  \begin{solutionordottedlines}[2cm]
    A afirmação é verdadeira pois o resultado da divisão destes números por 2 é um número inteiro (o resto é zero). 
  \end{solutionordottedlines}


  \question Efetua as operações seguintes:
  \begin{parts}
    \part[5] 1423 +1201
    \begin{solutionordottedlines}[1cm]
      2624
    \end{solutionordottedlines}
    \part[5] 432 -243
    \begin{solutionordottedlines}[1cm]
      189
    \end{solutionordottedlines}
    \part[10] 3521 x 45
    \begin{solutionordottedlines}[1cm]
    158445
  \end{solutionordottedlines}
  \part[15] 13 150 : 25
    \begin{solutionordottedlines}[1cm]
     526
    \end{solutionordottedlines}
  \end{parts}
  
\question[5] O Rui tem 12 anos e o seu pai tem mais 31 anos.
Qual é a idade do pai do Rui daqui a 8 anos?
\begin{solutionordottedlines}[1cm]
 31 + 8 = 39 anos    
 \end{solutionordottedlines}
    
 \question[10] A Joana comprou duas dúzias de maçãs, uma dezena de
laranjas e meia dúzia de peras. Quantas peças de fruta comprou a Joana?

\begin{solutionordottedlines}[1cm]
  $2 \times 12+10+6 = 40$
 \end{solutionordottedlines}


\question[15] A mãe da Matilde foi ao supermercado e comprou:
\begin{itemize}
  \item dois pacotes de sumo a 75 cêntimos cada um;
  \item cinco quilogramas de maçãs a 80 cêntimos cada um;
  \item um quilograma de bananas por 1 euro;
  \item nove quilogramas de batatas a 50 cêntimos cada um.
A mãe da Matilde pagou as suas compras com 20 euros. Quanto recebeu de troco?
\end{itemize}
\begin{solutionordottedlines}[2cm]
  $0.75 + 0.80 + 1 + 0.50 = 3.05$\\
  $20 - 3.05 = 16.95$ \qquad  euros
 \end{solutionordottedlines}
 

 \question[15] A Alice pensou num número. Multiplicou-o por dois. Ao valor obtido adicionou 54.
 O resultado foi 100. Em que número pensou a Alice?
 \begin{solutionordottedlines}[2cm]
   $\text{número} \times 2 + 54 = 100$\\
   número = (100-54) / 2 \\
   número = 46 / 2 = 23, a Alice pensou no número 23.
 \end{solutionordottedlines}

 
\end{questions}
    
\addpoints
\centering{\gradetable[h][questions]}


\end{document}